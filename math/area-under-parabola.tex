\font\eightrm=cmr8
{\eightrm The proof is excerpted from a textbook [1], I retyped it just for completeness
of this note.}

Slice the figure into a number of strips and obtain two
approximations to the region, one from below and one from
above, by using two sets of rectangles.
For the sake of simplicity, we subdivide the base into $n$ equal parts,
each of length $b/n$.  The points of subdivision correspond to the
following values of x: $$
0, {b\over n}, {2b\over n}, {3b\over n},\cdots, {(n-1)b\over n}, {nb\over n} = b.
$$
A typical point of subdivision corresponds to $x=kb/n$, where $k$ takes
the successive values $k=0,1,2,\ldots,n$. At each point $kb/n$ we construct
the outer rectangle of altitude $(kb/n)^2$. The area of this rectangle is the
product of its base and altitude and is equal to $$
\left({b\over n}\right)\left({kb\over n}\right)
={b^3\over n^3}k^2.
$$
Let us denote by $S_n$ the sum of the areas of all the outer rectangles.
Then since the the $k$th rectangle has area $(b^3/n^3)k^2$, we obtain the formula: $$
S_n={b^3\over n^3}(1^2+2^2+3^2+\cdots+n^2). \eqno(1)
$$
In the same way we obtain a formula for the sum $s_n$ of all the inner rectangles: $$
s_n={b^3\over n^3}(0^2+1^2+2^2+\cdots+(n-1)^2). \eqno(2)
$$
Before we go ahead, a formula should be taken out: $$
1^2+2^2+3^2+\cdots+n^2={n^3\over 3}+{n^2\over 2} + {n\over 6}. \eqno(3)
$$
This identity is valid for every integer $n\ge 1$ and can be proved as follows:
Start with the formula $(k+1)^3=k^3+3k^2+3k+1$ and rewrite it in the form: $$
3k^2+3k+1=(k+1)^3-k^3.
$$
Taking $k=1,2,\ldots,n-1$, we get $n-1$ formulas: $$
\eqalign{
3\cdot1^2+3\cdot1+1&=2^3-1^3\cr
3\cdot2^2+3\cdot2+1&=3^3-2^3\cr
\cdots\cr
3(n-1)^2+3(n-1)+1&=n^3-(n-1)^3.\cr
}
$$
When we add these formulas, all the terms on the right cancel except two and we obtain $$
3\bigl(1^2+2^2+\cdots+(n-1)^2\bigr)+
3\bigl(1+2+\cdots+(n-1)\bigr)+\bigl(n-1\bigr)=n^3-1^3.
$$
The second sum on the left is the sum of terms in an arithmetic progression and it
simplifies to ${1\over2}n(n-1)$.  Therefore this last equation gives us: $$
1^2+2^2+\cdots+(n-1)^2={n^3\over3}-{n^2\over2}+{n\over6}. \eqno(4)
$$
Adding $n^2$ to both members, we obtain (3).

For our purposes, we do not need the exact expressions given in the right-hand members
of (3) and (4).  All we need are the two inequalities: $$
1^2+2^2+\cdots+(n-1)^2<{n^3\over 3}<1^2+2^2+\cdots+n^2 \eqno(5)
$$
which are valid for every integer $n\ge 1$. These inequalities can be deduced easily as
consequences of (3) and (4), or they can be proved directly by induction.

If we multiply both inequalities in (5) by $b^3/n^3$ and make use of (1) and (2), we obtain $$
s_n<{b^3\over 3}<S_n \eqno(6)
$$
for every n. The inequalities in (6) tell us that $b^3/3$ is a number which lies between $s_n$
and $S_n$ for every $n$.  We will now prove that $b^3/3$ is the only number which has
this property.  In other words, we assert that if $A$ is any number which satisfies the
inequalities $$
s_n<A<S_n \eqno(7)
$$
for every positive integer $n$, then $A=b^3/3$.

To prove that $A=b^3/3$, we use the inequalities in (5) once more.  Adding $n^2$ to both
sides of the leftmost inequality in (5), we obtain $$
1^2+2^2+\cdots+n^2<{n^3\over3}+n^2.
$$
Multiplying this by $b^3/n^3$ and using (1), we find $$
S_n<{b^3\over 3} + {b^3\over n}. \eqno(8)
$$
Similarly, by substracting $n^2$ from both sides of the rightmost inequality in (5)
and multiplying by $b^3/n^3$, we are led to the inequality $$
{b^3\over 3}-{b^3\over n} < s_n. \eqno(9)
$$
Therefore, any number $A$ satisfying (7) must also satisfy $$
{b^3\over 3}-{b^3\over n} < A < {b^3\over 3} + {b^3\over n} \eqno(10)
$$
for every integer $n\ge 1$.  Now there are only three possibilities: $$
A>{b^3\over 3}, A<{b^3\over 3}, A={b^3\over 3}.
$$
If we show that each of the first two leads to a contradiction, then we
must have $A={b^3\over 3}$, since this exhausts all the possibilities.

Suppose the inequality $A>{b^3\over3}$ were true.  From the second inequality in (10)
we obtain $$
A-{b^3\over 3} < {b^3\over n} \eqno(11)
$$
for every integer $n\ge 1$.  Since $A-{b^3\over 3}$ is positive, we may divide both sides of (11)
by $A-{b^3\over 3}$ and multiply by $n$ to obtain the equivalent statement $$
n < {b^3\over A-b^3/3}
$$
for every $n$.  But this inequality is obviously false when $n\ge b^3/(A-b^3/3)$.
Hence the inequality $A>b^3/3$ leads to a contradiction.  By a similar argument,
we can show that the inequality $A<b^3/3$ also leads to a contradiction, and therefore
we must have $A=b^3/3$, as asserted.

{\eightrm The following is my thoughts.}

Proofs, even tell a same thing, have different qualities.  The proof showed above
is well-formed in my opinion for it decoupled $1^2+2^2+\cdots+n^2$ from $b^3/n^3$ but
do not do so for $b^3$ and $n^3$.
\end

