Recall the intro part of integral
in Calculus' class.
If you were the same as me,
you mignt have faced a concept naming {\it Riemann sum}.
We expanded finite Riemann sums to infinite,
then the integrals began.
Here we go in the opposite direction, 
try to find out the counterpart of differentials in summation.

To make a presentation, 
let $y=f(x)$, then:
$$
dy = f'(x)\,dx\eqno(1)
$$

And attempt to find the corresponding in a sum $S_n = \sum_{k=1}^na_k$,
like $x$ in $f'(x)$, 
we need to specify the index $k$,
then, the answer is clear:
$$
\eqalign{
\Delta S_k &= S_{k+1} - S_k \cr
&= a_{k+1}\cr
}\eqno(2)
$$
Denoted by $\Delta$.

This is the critical point, 
and the reason why I write this essay
despite the outstanding material
provided already by Knuth's {\it Concrete Mathematics}.
Let me point it out more precisely, 
why dose $\Delta S_k$ equal to $a_{k+1}$, not $a_k$?

If we considered the $\Delta S_k$ simply as
the elements to be summed up, 
because of the definition of $S_n$,
it was naturally equated to $a_k$.
But, the meaning behind $\Delta$ is increment,
and the $k$th increment is much better
to be considered as
the next increment put on $k$th $S_n$(namely $S_k$),
computed by $S_{k+1} - S_k$, 
what we have seen before at $(2)$.

There's a innate difference between differentials and 
the terms included in a summation, 
one need a base, and one needn't. 

I was confused for a long time about the indexes 
when trying to understand {\it Finite Calculus}, 
this essay is wrote as a reminder for myself.

